\documentclass[10pt,spanish,aspectratio=1610]{beamer}
\usepackage[utf8]{inputenc}
\usepackage{amsmath}
\usepackage{graphicx}
\usepackage{amssymb}
\usepackage[spanish]{babel}
\spanishdecimal{.}
\usepackage{subfig}
\usepackage{fancyhdr}
\usepackage{pstricks}
\usepackage{color}
\usepackage[ruled]{algorithm2e}
\usepackage{listings}
\usepackage{multicol}
\usepackage{xcolor}

\definecolor{graywhite}{rgb}{0.9529,0.9607,0.9686}
\definecolor{bluegray}{rgb}{0.6823, 0.7411, 0.8}
\definecolor{darkred}{rgb}{0.7372, 0.2392, 0.2392}
\definecolor{bluedark}{rgb}{0.294, 0.4705, 0.6407}
\definecolor{darkgreen}{rgb}{0.1764, 0.5294, 0.0901}
\lstset{
  backgroundcolor=\color{graywhite},   % choose the background color; you must add \usepackage{color} or \usepackage{xcolor}; should come as last argument
  basicstyle=\footnotesize,        % the size of the fonts that are used for the code
  breakatwhitespace=false,         % sets if automatic breaks should only happen at whitespace
  breaklines=true,                 % sets automatic line breaking
  captionpos=b,                    % sets the caption-position to bottom
  commentstyle=\color{darkgreen},    % comment style
  keepspaces=true,                 % keeps spaces in text, useful for keeping indentation of code (possibly needs columns=flexible)
  keywordstyle=\color{darkred},       % keyword style
  language=Octave,                 % the language of the code
  morekeywords={*,...},            % if you want to add more keywords to the set
  numbers=left,                    % where to put the line-numbers; possible values are (none, left, right)
  numbersep=7pt,                   % how far the line-numbers are from the code
  numberstyle=\tiny\color{bluedark}, % the style that is used for the line-numbers
  showspaces=false,                % show spaces everywhere adding particular underscores; it overrides 'showstringspaces'
  showstringspaces=false,          % underline spaces within strings only
  showtabs=false,                  % show tabs within strings adding particular underscores
  stepnumber=1,                    % the step between two line-numbers. If it's 1, each line will be numbered
  stringstyle=\color{bluedark},     % string literal style
  frame=single,
  rulecolor=\color{bluegray},
  tabsize=2,                   % sets default tabsize to 2 spaces
  xleftmargin=1cm,
  xrightmargin=0.5cm,
  framexleftmargin=0.5cm,
  extendedchars=true,
  literate={á}{{\'a}}1 {é}{{\'e}}1 {í}{{\'i}}1 {ó}{{\'o}}1 {ú}{{\'u}}1 {Á}{{\'A}}1 {É}{{\'E}}1 {Í}{{\'I}}1 {Ó}{{\'O}}1 {Ú}{{\'U}}1,
}

\DeclareMathOperator{\atantwo}{atan2}
\setbeamercolor{block title}{fg=white,bg=blue!70!black}
\setbeamercolor{block body}{fg=black, bg=blue!10!white}
\setbeamertemplate{blocks}[rounded][shadow=false]
\setbeamercovered{transparent}
\beamertemplatenavigationsymbolsempty
\setbeamertemplate{frametitle}{
  \leavevmode
  \hbox{\begin{beamercolorbox}[wd=0.6\paperwidth,left]{frametitle}
    \usebeamerfont{frametitle}\insertframetitle
  \end{beamercolorbox}
  \begin{beamercolorbox}[wd=0.4\paperwidth,center]{frametitle}
    \usebeamerfont{frametitle}\small{\thesection . \insertsectionhead}
  \end{beamercolorbox}  } }
\setbeamertemplate{footline}{
  \leavevmode%
  \hbox{%
    \begin{beamercolorbox}[colsep=-0.5pt,wd=.33\paperwidth,ht=3ex,dp=1.5ex,center]{author in head/foot}%
      \usebeamerfont{author in head/foot}\insertshortauthor~~ (\insertshortinstitute)
    \end{beamercolorbox}%
    \begin{beamercolorbox}[colsep=-0.5pt,wd=.34\paperwidth,ht=3ex,dp=1.5ex,center]{date in head/foot}%
      \usebeamerfont{author in head/foot}\insertshorttitle
    \end{beamercolorbox}%
    \begin{beamercolorbox}[colsep=-0.5pt,wd=.33\paperwidth,ht=3ex,dp=1.5ex,right]{author in head/foot}%
      \usebeamerfont{author in head/foot}\insertshortdate{} \hspace*{2em}\scriptsize{\insertframenumber{}}\hspace*{1ex}
    \end{beamercolorbox}
  }
}

\begin{document}
\renewcommand{\tablename}{Tabla}
\renewcommand{\figurename}{Figura}

\title[Tareas Básicas en Robots de Servicio Doméstico]{Tareas Básicas en Robots de Servicio Doméstico}
\author[Marco Negrete]{Instructor: Marco Antonio Negrete Villanueva}
\institute[FI, UNAM]{Facultad de Ingeniería, UNAM}
\date[EIR 2022]{Escuela de Invierno de Robótica 2022\\Modalidad a distancia\\\url{https://github.com/mnegretev/EIR-2022-AtHomeBasicTasks}}

\begin{frame}
\titlepage
\end{frame}

\begin{frame}
  \Large{Objetivos:}
  \normalsize
  \[\]
  \textbf{Objetivo General:} Brindar los conocimientos básicos necesarios para desarrollar un robot de servicio doméstico.
  \\
  \textbf{Objetivos Específicos:}
  \begin{itemize}
  \item Revisar el hardware necesario para tener un robot de servicio doméstico: sensores y actuadores más comunes.
  \item Dar un panorama general del software necesario para desarrollar un robot de servicio doméstico.
  \item Revisar las herramientas disponibles para cubrir las habilidades básicas requeridas en un robot de servicio doméstico:
    \begin{itemize}
    \item Navegación
    \item Vision
    \item Manipulación
    \item Síntesis y reconocimiento de voz
    \item Planeación de acciones
    %\item MoveIt (para manipulación de objetos)
    \end{itemize}
  \end{itemize}
\end{frame}

\begin{frame}
  \Large{Contenido}
  \normalsize
  \[\]

  \tableofcontents
\end{frame}


\section{Introducción}

\begin{frame}
  \Huge
  Introducción
\end{frame}

\begin{frame}\frametitle{Los robots de servicio doméstico}
  \begin{columns}
    \begin{column}{0.5\textwidth}
      Son robots pensados para ayudar en tareas comunes del hogar u oficina. Requieren de varias habilidades:
      \begin{itemize}
      \item Interacción humano-robot
      \item Navegación en ambientes dinámicos
      \item Reconocimiento de objetos
      \item Manipulación de objetos
      \item Comportamientos adaptables
      \item Planeación de acciones
      \end{itemize}
    \end{column}
    \begin{column}{0.4\textwidth}
      \centering
      \includegraphics[width=0.9\textwidth]{Figures/Justina.pdf}
    \end{column}
  \end{columns}
\end{frame}

\begin{frame}\frametitle{Hardware necesario: Base móvil}
  \begin{columns}
    \begin{column}{0.5\textwidth}
      \begin{itemize}
      \item De preferencia, debe ser omnidireccional
      \item Turtle Bot (\url{https://www.turtlebot.com/})
      \item Festo Robotino (\url{https://wiki.openrobotino.org/})
      \item DIY: 3 ó 4 motores de corriente directa con ruedas omnidireccionales, 2 tarjetas Roboclaw, baterías de LiPo y chasis de alumnio estructural.
      \end{itemize}
    \end{column}
    \begin{column}{0.5\textwidth}
      \includegraphics[width=\textwidth]{Figures/Bases.png}
    \end{column}
  \end{columns}
\end{frame}

\begin{frame}\frametitle{Hardware necesario: Cámaras}
  \begin{columns}
    \begin{column}{0.45\textwidth}
      \includegraphics[width=\textwidth]{Figures/Cameras.jpg}
    \end{column}
    \begin{column}{0.5\textwidth}
      \begin{itemize}
      \item Se pueden usar sólo cámaras RGB, pero es altamente recomendable tener información de profundidad.
      \item Kinect (\url{https://github.com/OpenKinect/libfreenect2})
      \item Intel RealSense (\url{https://github.com/IntelRealSense/librealsense})
      \item También se pueden usar cámaras estéreo, pero es mucho más sencillo usar cámaras con luz estructurada.
      \end{itemize}
    \end{column}
  \end{columns}
\end{frame}

\begin{frame}\frametitle{Hardware necesario: Sensor láser}
  \begin{columns}
    \begin{column}{0.5\textwidth}
      \begin{itemize}
      \item Hokuyo (\url{https://www.hokuyo-aut.jp/})
      \item RPLidar (\url{https://www.robotshop.com/en/slamtec.html})
      \item SICK (\url{https://www.sick.com/ag/en/detection-and-ranging-solutions/2d-lidar-sensors/c/g91900})
      \item El paquete \url{http://wiki.ros.org/urg_node} facilita su operación.
      \item Si no se tiene uno, se puede simular a partir de una cámara RGB-D con el paquete \url{http://wiki.ros.org/pointcloud_to_laserscan}.
      \end{itemize}
    \end{column}
    \begin{column}{0.4\textwidth}
      \includegraphics[width=\textwidth]{Figures/lasers.jpg}
    \end{column}
  \end{columns}
\end{frame}

\begin{frame}\frametitle{Hardware necesario: Manipulador}
  \begin{columns}
    \begin{column}{0.4\textwidth}
      \includegraphics[width=\textwidth]{Figures/arms.jpg}
    \end{column}
    \begin{column}{0.5\textwidth}
      \begin{itemize}
      \item Son recomendables por lo menos 5 DOF.
      \item Kuka LBR iiwa (\url{http://wiki.ros.org/kuka})
      \item Neuronics Katana (\url{http://wiki.ros.org/katana})
      \item DIY: Servomotores y Brackets Dynamixel (\url{http://wiki.ros.org/dynamixel})
      \end{itemize}
    \end{column}
  \end{columns}
\end{frame}

\section{ROS}

\begin{frame}
  \Huge
  La plataforma ROS
\end{frame}


\begin{frame}\frametitle{La plataforma ROS}
  \includegraphics[width=0.3\textwidth]{Figures/Ros_logo.png}
  \[\]
  \textbf{ROS (Robot Operating System) } es un \textit{middleware} de código abierto para el desarrollo de robots móviles.
  \begin{itemize}
  \item Implementa funcionalidades comúnmente usadas en el desarrollo de robots como el paso de mensajes entre procesos y la administración de paquetes.
  \item Muchos drivers y algoritmos ya están implementados.
  \item Es una plataforma distribuida de procesos (llamados \textit{nodos}).
  \item Facilita el reuso de código.
  \item Independiente del lenguaje (Python y C++ son los más usados).
  \item Facilita el escalamiento para proyectos de gran escala. 
  \end{itemize}
\end{frame}

\begin{frame}\frametitle{Conceptos}
  ROS se puede entender en dos grandes niveles conceptuales:
  \begin{itemize}
  \item \textbf{Sistema de archivos:} Recursos de ROS en disco
  \item \textbf{Grafo de procesos:} Una red \textit{peer-to-peer} de procesos (llamados nodos) en tiempo de ejecución.
  \end{itemize}
\end{frame}

\begin{frame}\frametitle{Sistema de archivos}
  \begin{columns}
    \begin{column}{0.5\textwidth}
      Recursos en disco:
      \begin{itemize}
      \item \textbf{Workspace:} carpeta que contiene los paquete desarrollados
      \item \textbf{Paquetes:} Principal unidad de organización del software en ROS (concepto heredado de Linux)
      \item \textbf{Manifiesto:} (\texttt{package.xml}) provee metadatos sobre el paquete (dependencias, banderas de compilación, información del desarrollador)
      \item \textbf{Mensajes (msg):} Archivos que definen la estructura de un \textit{mensaje} en ROS.
        \item \textbf{Servicios (srv):} Archivos que definen las estructuras de la petición (\textit{request}) y respuesta (\textit{response}) de un servicio. 
      \end{itemize}
    \end{column}
    \begin{column}{0.4\textwidth}
      \includegraphics[width=\textwidth]{Figures/catkin_tree.png}
    \end{column}
  \end{columns}
\end{frame}

\begin{frame}\frametitle{Grafo de procesos}
  El grafo de procesos es una red \textit{peer-to-peer} de programas (nodos) que intercambian información entre sí. Los principales componentes del este grafo son:
  \[\]
  \begin{columns}
    \begin{column}{0.5\textwidth}
      \begin{itemize}
      \item master
      \item servidor de parámetros
      \item nodos
      \item mensajes
      \item servicios 
      \end{itemize}
    \end{column}
    \begin{column}{0.5\textwidth}
      \includegraphics[width=\textwidth]{Figures/RosGraph.pdf}
    \end{column}
  \end{columns}
\end{frame}

\begin{frame}\frametitle{Tópicos y servicios}
  Los nodos (procesos) en ROS intercambian información a través de dos grandes patrones:
  \[\]
  \begin{columns}
    \begin{column}{0.6\textwidth}
        \begin{itemize}
        \item \textbf{Tópicos}
          \begin{itemize}
          \item Son un patrón $1:n$ de tipo \textit{publicador/suscriptor}
          \item Son no bloqueantes
          \item Utilizan estructuras de datos definidas en archivos \texttt{*.msg} para el envío de información
          \end{itemize}
        \item \textbf{Servicios}
          \begin{itemize}
          \item Son un patrón $1:1$ de tipo \textit{petición/respuesta}
          \item Son bloqueantes
          \item Utilizan estructuras de datos definidas en archivos \texttt{*.srv} para el intercambio de información. 
          \end{itemize}
        \end{itemize}
    \end{column}
    \begin{column}{0.4\textwidth}
      \includegraphics[width=\textwidth]{Figures/RosGraph.pdf}
    \end{column}
  \end{columns}
  \[\]
  Para mayor información:
  \begin{itemize}
  \item Tutoriales \url{http://wiki.ros.org/ROS/Tutorials}
  \item Koubâa, A. (Ed.). (2020). Robot Operating System (ROS): The Complete Reference. Springer Nature
  \end{itemize}
\end{frame}

%\section{Conceptos Básicos}

\begin{frame}
  \Huge
  Conceptos básicos
\end{frame}

\begin{frame}\frametitle{Funciones comunes}
  \textbf{Sigmoide}: Es una función que puede ser usada como una versión \textit{suave} del escalón. Se usará en en control de posición y en el entrenamiento de redes neuronales.
  \[\sigma(x) = \frac{1}{1 + e^{-x}}\]
  La derivada tiene la forma:
  \[\frac{d\sigma}{dx}=\frac{-(-e^{-x})}{\left(1 + e^{-x}\right)^2} = \frac{1 + e^{-x} - 1}{\left(1 + e^{-x}\right)^2}=\frac{1}{1+e^{-x}}\left(1 - \frac{1}{1 + e^{-x}}\right) = \sigma(x)(1 - \sigma(x))\]
  \textbf{Campana de Gauss}: Es una función siempre positiva que tiende a cero cuando $|x|\rightarrow \infty$:
  \[f(x) = a \cdot e^{-\frac{(x - b)^2}{2c^2}}\]
\end{frame}

\begin{frame}\frametitle{Gradiente}
  Dada una función $f:\mathbb{R}^n \rightarrow \mathbb{R}$, es decir, una función escalar de variable vectorial $f(x_1, x_2, \dots, x_n)$, el gradiente $\nabla f(\bar{x})$ está dado por:
  \[\nabla f(\bar{x}) = \left[ \frac{\partial f}{\partial x_1}, \frac{\partial f}{\partial x_2}, \dots \frac{\partial f}{\partial x_n}\right]^T\]

  \begin{itemize}
  \item El gradiente generaliza el concepto de derivada para funciones de varias variables.
  \item El gradiente evaluado en un punto $\bar{x}_0$ indica la dirección de máximo cambio en ese punto.
  \end{itemize}
\end{frame}

\begin{frame}\frametitle{Jacobiano}
  Dada una función $F:\mathbb{R}^n \rightarrow \mathbb{R}^m$, es decir, una función vectorial de variable vectorial:
  \[F(\bar{x}) = \begin{tabular}{l}$f_1(x_1, x_2, ..., x_n)$\\$f_2(x_1, x_2, ..., x_n)$\\ $\vdots$ \\ $f_n(x_1, x_2, ..., x_n)$\end{tabular}\]
  El Jacobiano es una matriz que contiene las primeras derivadas parciales:
  \[J(x) = \left[\begin{tabular}{cccc}
      $\dfrac{\partial f_1}{\partial x_1}$ & $\dfrac{\partial f_1}{\partial x_2}$ & $\dots$ & $\dfrac{\partial f_1}{\partial x_n}$\\
      & & &\\
      $\dfrac{\partial f_2}{\partial x_1}$ & $\dfrac{\partial f_2}{\partial x_2}$ & $\dots$ & $\dfrac{\partial f_2}{\partial x_n}$\\
      & $\vdots$ & & \\
      $\dfrac{\partial f_m}{\partial x_1}$ & $\dfrac{\partial f_m}{\partial x_2}$ & $\dots$ & $\dfrac{\partial f_m}{\partial x_n}$\\
    \end{tabular}\right] = \left[\begin{tabular}{c}$\nabla^T f_1$\\ $\nabla^T f_2$ \\ $\vdots$ \\ $\nabla^T f_m$\end{tabular}\right]\in\mathbb{R}^{m\times n}\]
\end{frame}

\begin{frame}\frametitle{Espacio de Configuraciones}
  La \textit{configuración} de un robot (o de una parte de él) es una descripción de todos los puntos que ocupa en el espacio. Los \textit{grados de libertad} son el conjunto mínimo de valores independientes que se requieren para definir una configuración.
  \[\]
  La configuración de un cuerpo rígido que se mueve en el espacio tiene 6 grados de libertad (tres para posición y tres para orientación): $(x,y,z,\psi,\theta,\phi)$. Si el cuerpo se mueve sólo en el plano, tiene 3 grados: $(x,y,\theta)$ (una posición en 2D y una orientación).
\end{frame}


\section[P. de movimientos]{Planeación de movimientos}

\begin{frame}\frametitle{Planeación de movimientos}
  El problema de la planeación de movimientos comprende cuatro tareas básicas:
  \begin{itemize}
  \item Navegación (encontrar una ruta por el espacio libre de un punto inicial a uno final). Si la ruta está parametrizada con repecto al tiempo, se dice que es una trayectoria.
  \item Mapeo (construir una representación del ambiente a partir de las lecturas de los sensores y la configuración)
  \item Localización (determinar la configuración a partir de un mapa y de lecturas de los sensores)
  \item Barrido (pasar un actuador por todos los puntos de un subespacio)
  \end{itemize}
  Comúnmente el mapeo y la localización se realizan al mismo tiempo en el proceso conocido como SLAM (\textit{Simultaneous Localization and Mapping})
\end{frame}

\begin{frame}\frametitle{Celdas de ocupación}
  Es una discretización del espacio con una resolución determinada donde a cada celda se le asigna un número $p\in[0,1]$ que indica su nivel de ocupación.
  \begin{itemize}
  \item En un enfoque probabilístico, $p$ indica la certeza de que la celda esté ocupada: 0, certeza de que está libre, 1, certeza de que está ocupada, 0.5, no se tiene información.
  \item En este curso, los niveles de ocupación solo serán 0 o 1.
 Para evitar el manejo de flotantes, el nivel de ocupación suele representarse con un entero en el intervalo [0,100] y un -1 si no hay información. 
  \end{itemize}
  \includegraphics[width=0.4\textwidth]{Figures/OccupancyGrid.png}
  \includegraphics[width=0.4\textwidth]{Figures/OccupancyGridZoom.png}
\end{frame}

\begin{frame}[containsverbatim]\frametitle{Ejercicio}
  Ejecute el comando:
  \begin{lstlisting}
    roslaunch bring_up path_planning.launch
  \end{lstlisting}
  Inspeccione el mapa en el visualizador RViz. Luego abra el archivo:
  \texttt{catkin\_ws/src/config\_files/maps/appartment.pgm}
  con cualquier editor de imágenes y modifíquelo.
  \\Detenga la ejecución y vuelva a correr el comando anterior. 
\end{frame}

\begin{frame}\frametitle{Planeación de rutas}
  Dado un espacio de configuraciones $Q$ con espacio libre $Q_{free}\subset Q$ y espacio ocupado $Q_{occ}\subset Q$, la planeación de rutas consiste en contrar un mapeo:
  \[f: [0,1] \rightarrow Q_{free} \qquad \textrm{con}\qquad f(0) = q_s\qquad f(1)=q_g\]
  donde $q_s$ y $q_g$ son las configuraciones inicial y meta, respectivamente. Es decir, se debe encontrar una secuencia de puntos del espacio libre que permitan al robot moverse del punto inicial al punto meta sin chocar. Los métodos para planear rutas se pueden agrupar en:
  \begin{itemize}
  \item Basados en búsqueda en grafos (A*, Dijkstra)
    \begin{itemize}
    \item En un mapa de celdas de ocupación, cada celda libre es un nodo del grafo.
    \item Cada nodo está conectado con las celdas vecinas del espacio libre. 
    \end{itemize}
  \item Basados en muestreo (RRT)
  \item Variacionales
  \end{itemize}
\end{frame}

\begin{frame}\frametitle{El algoritmo A*}
    \begin{algorithm}[H]
    \footnotesize
    \DontPrintSemicolon
    \KwData {Mapa $M$ de celdas de ocupación, configuración inicial $q_{start}$, configuración meta $q_{goal}$}
    \KwResult{Ruta $P=[q_{start},q_1, q_2, \dots , q_{goal}]$}
    \;
    Obtener los nodos $n_s$ y $n_g$ correspondientes a $q_{start}$ y $q_{goal}$\;
    Lista abierta $OL = \emptyset$ y lista cerrada $CL = \emptyset$\;
    \ForAll{Nodo $n \in M$}
    {
      $g(n) = \infty \qquad f(n) = \infty$\;
    }
    Agregar $n_s$ a $OL$\;
    $g(n_s) = 0 \qquad f(n_s) = 0$\;
    Nodo actual $n_c = n_s$\;
    \While{$OL\neq \emptyset$ y $n_c\neq n_g$}
    {
      Seleccionar de $OL$ el nodo $n_c$ con el menor valor $f$ y agregar $n_c$ a $CL$\;
      \ForAll{Vecino $n$ de $n_c$}
      {
        Calcular los valores $g$ y $h$ para el nodo $n$\;
        \If{$g < g(n)$}
        {
          $g(n) = g\qquad f(n) = g + h \qquad Previo(n) = n_c$
        }
      }
      Agregar a $OL$ los vecinos de $n_c$ que no estén ya en $OL$ ni en $CL$\;
    }
    \If{$n_c\neq n_g$}{Anunciar Falla}
    Obtener la configuración $q_i$ para cada nodo $n_i$ de la ruta\;
  \end{algorithm}
\end{frame}

\begin{frame}\frametitle{El algoritmo A*}
  El valor $g$ es una función de costo y el algoritmo A* siempre devolverá la ruta que minimice el costo total del nodo inicial al nodo meta. Las funciones más comunes son:
  \begin{itemize}
  \item Distancia de Manhattan: $d(p_1, p_2) = |p_{1_x} - p_{2_x}| + |p_{1_y} - p_{2_y}|$
  \item Distancia Euclidiana: $d(p_1, p_2) = \left( (p_{1_x} - p_{2_x})^2 + (p_{1_y} - p_{2_y})^2 \right)^{1/2}$
  \end{itemize}
  En la función $g$ se puede incluir cualquier criterio para planear una ruta: la más corta, la más rápida, la más segura, etc.\\
  La heurística $h$ sirve para expandir menos nodos y es una función que debe \textit{subestimar} el costo real de llegar de un nodo $n$ al nodo meta $n_g$. La distancia Euclidiana y la distancia de Manhattan son dos funciones también muy usadas como heurísticas. 
\end{frame}

\begin{frame}[containsverbatim]\frametitle{Ejercicio}
  Modificar el archivo a\_star.py para cambiar $g$ y $h$ de Manhattan a Euclidiana.
  \texttt{catkin\_ws/src/config\_files/maps/appartment.pgm}
    \lstinputlisting[language=Python]{Codes/AStar.py}
\end{frame}



\section{Visión}
\begin{frame}\frametitle{Las imágenes como funciones}
Una imagen (en escala de grises) es una función $I(x,y)$ donde $x,y$ son variables discretas en coordenadas de imagen y la función $I$ es intensidad luminosa.

  Las imágenes también pueden considerarse como arreglos bidimensionales de números entre un mínimo y un máximo (usualmente 0-255).

  Aunque formalmente una imagen es un mapeo $f:\mathbb{R}^2\rightarrow \mathbb{R}$, en la práctica, tanto $x,y$ como $I$ son varialbes discretas con valores entre un mínimo y un máximo.

  Las imágenes de color son funciones vectoriales $f:\mathbb{R}^2\rightarrow \mathbb{R}^3$ donde cada componente de la función se llama canal:
  \[I(x,y) = \left[\begin{tabular}{c}$r(x,y)$\\$g(x,y)$\\$b(x,y)$\end{tabular}\right]\]  
\end{frame}

\begin{frame}\frametitle{Espacios de color}
  
\end{frame}

\begin{frame}\frametitle{Histograma de color}
\end{frame}

\begin{frame}\frametitle{Nubes de puntos}
  
\end{frame}

\begin{frame}\frametitle{Extracción de planos}
  
\end{frame}


\begin{frame}\frametitle{Agrupamiento}
  
\end{frame}

\begin{frame}\frametitle{Ejercicio}
Quitar plano de la mesa y reconocer por color.   
\end{frame}

\begin{frame}\frametitle{Redes neuronales}
  
\end{frame}


\begin{frame}\frametitle{Ejercicio}
  
\end{frame}


\section{Manipulación}
\begin{frame}\frametitle{Posición y Orientación}
  Un cuerpo rígido en el espacio puede tener una posición $(x,y,z)$ y una orientación. La orientación se puede representar de varias formas:
  \begin{itemize}
  \item Mediante ángulos de Euler: roll, pich y yaw $RPY = (\psi, \theta, \phi)$
  \item Mediante cuaterniones
  \item Mediante una matriz de rotación $R \in SO(3)$
  \end{itemize}
  Los ángulos $RPY$ son rotaciones intrínsecas sobre los ejes $X$, $Y$, y $Z$ respectivamente. Se llaman intrínsecas porque son rotaciones que ocurren sobre un sistema de referencia \textit{atado} a un cuerpo rígido.\\
  Cualquier orientación se puede obtener mediante la composición de tres rotaciones básicas:
  \[R = R_{z,\phi}R_{y,\theta}R_{x,\psi}\]
  Es decir, primero una rotación de $\phi$ radianes sobre el eje $Z$, seguida de una rotación de $\theta$ radianes sobre el eje $Y$ del sistema resultante y una rotación de $\psi$ radianes sobre el eje $X$ del sistema rotado. 
\end{frame}

\begin{frame}\frametitle{Transformaciones Homogéneas}
  Una Transformación Homogénea es una matriz de la forma
  \[T = \left[\begin{tabular}{cccc}
      & & & $d_x$\\
      & $R\in SO(3)$ & & $d_y$\\
      & & & $d_z$\\
      0 & 0& 0 & 1
    \end{tabular}\right]\]
  Puede servir para
  \begin{itemize}
  \item Representar la posición y orientación de un cuerpo rígido
  \item Representar una transformación de coordenadas $T_{ab}$ de un sistema de referencia $b$ a un sistema $a$
  \end{itemize}
  Propiedades:
  \begin{itemize}
  \item Asociatividad: $(T_1 T_2) T_3 = T_1 (T_2 T_3)$
  \item Inversa:
    \[T = \left[\begin{tabular}{cc}
       $R^T$ & $-R^T d$\\
       0 & 1
      \end{tabular}\right]\]
  \item Cancelación de índices: $T_{ab} = T_{ac}T_{cb}$
  \end{itemize}
\end{frame}

\begin{frame}\frametitle{El árbol cinemático}
  Es útil tener una descripción de la forma en que están conectadas las diferentes articulaciones del robot. Se considera que sobre cada articulación hay un sistema de referencia (\textit{frame}) que está trasladado y rotado con respecto al sistema anterior.
  \begin{multicols}{3}
    \includegraphics[width=0.26\textwidth]{Figures/KinematicTree.png}\\
    \footnotesize
    El sistema \textit{absoluto} se suele llamar \texttt{map}\\
    El sistema base del robot se suele llamar \textit{base\_link}\\
    Las transformaciones de \texttt{map} a \texttt{base\_link} las determina el sistema de localización\\
    El resto de las transformaciones se determinan con la posición de cada articulación\\
    El árbol cinemático se traduce en una cadena de mulplicaciones de Transformaciones Homogéneas. 
    \includegraphics[width=0.35\textwidth]{Figures/TfTree.pdf}
  \end{multicols}
\end{frame}

\begin{frame}[containsverbatim]\frametitle{El formato URDF}
  El formato URDF permite describir el arbol cinemático del robot mediante etiquetas XML:
  \footnotesize
  \lstinputlisting[language=XML]{Codes/URDFExample.xml}
  \normalsize
  Cada etiqueta \texttt{<joint>} representará una Transformación Homogénea. 
\end{frame}

\begin{frame}[containsverbatim]\frametitle{El formato Xacro}
  El formato Xacro es un lenguaje de \textit{macros} que permite obtener archivos XML más cortos. Es últil para especificar parámetros físicos en el URDF como inercias y volúmenes:
  \footnotesize
  \lstinputlisting[language=XML]{Codes/XacroExample.xml}
\end{frame}

\begin{frame}[containsverbatim]\frametitle{Ejercicio}
  Abra el archivo \texttt{catkin\_ws/src/hardware/justina\_description/urdf/justina\_base.xacro} y vaya a la línea 228:
  \lstinputlisting[language=XML,firstnumber=227]{Codes/JustinaXacro.xml}
  Modifique el atributo \texttt{xyz} y aumente 1 m en la coordenada en z. Después ejecute el comando:
  \begin{lstlisting}
    roslaunch bring_up path_planning.launch
  \end{lstlisting}
  Detenga la simulación. Ahora modifique el atributo \texttt{rpy}, cambie los valores a ``1.5708 0 0'' y ejecute de nuevo la simulación. 
\end{frame}

\begin{frame}\frametitle{La cinemática directa}
  La cinemática directa consiste en determinar la posición y orientación del efector final del manipulador a partir de la posición de cada articulación.  Esta se puede calcular con la ecuación:
  \[P_1 = T_{12}T_{23}T_{34}T_{45}T_{56}T_{67}T_{7g}P_g\]
  donde $P_g = [0,0,0,1]^T$ es la posición del gripper con respecto al sistema del gripper, $P_1$ es la posición del gripper con respecto al sistema base y $T_{ab}$ es la transformación homogénea que define la rotación y traslación producida por cada articulación. Las matrices $T_{ab}$
  tienen la forma:
  \[T_{ab} = \left[\begin{tabular}{cccc}
      & & & $dx_{ab}$\\
      & $R_{ab}\in SO(3)$ & & $dy_{ab}$\\
      & & & $dz_{ab}$\\
      0 & 0& 0 & 1
    \end{tabular}\right]\]
  Donde $R_{ab}$ representa la rotación del sistema $b$ respecto al sistema $a$ y $(dx_{ab}, dy_{ab}, dz_{ab})$ es la traslación del sistema $b$ respecto al sistema $a$.\\
  La rotación $R_{ab}$ está definida en el URDF por el atributo ``rpy'' de la sub etiqueta \texttt{origin} de la etiqueta \texttt{joint} y por la posición de la articulación. La traslación $(dx_{ab}, dy_{ab}, dz_{ab})$ está definida por el atributo ``xyz''. 
\end{frame}

\begin{frame}\frametitle{Ejercicio}
  Abra el archivo ... y modifique las líneas ...
\end{frame}

\begin{frame}\frametitle{La cinemática inversa}
  La cinemática inversa consiste en determinar las posiciones que debe tener cada articulación para que el efector final tenga la posición y orientación deseadas.
  \begin{itemize}
  \item Mientras la cinemática directa siempre tiene solución, la cinemática inversa, no.
  \item Se puede resolver por métodos geométricos para obtener una solución cerrada, aunque el análisis puede ser muy complicado.
  \item Una solución más general se puede obtener mediante un método numérico. 
  \end{itemize}
  Suponiendo que se tiene una configuración deseada $p_d \in \mathbb{R}^6$ ($xyz-RPY$), se desea encontrar el conjunto de posiciones articulares $q\in\mathbb{R}^7$ que satisfagan la ecuación
  \[FK(q) - p_d = 0\]
  donde la función $FK$ representa la cinemática directa. 
\end{frame}

\begin{frame}\frametitle{El método Newton-Raphson}
  El método numérico de Newton-Raphson sirve para encontrar raíces, es decir, para resolver ecuaciones de la forma
  \[f(x) = 0\]
  El algoritmo es el siguiente:
  \begin{algorithm}[H]
    \DontPrintSemicolon
    $x \leftarrow x_0$\;
    \While{$|f(x)| > \epsilon$}
    {
       $x \leftarrow x - \frac{f(x)}{f'(x)}$
    }
  \end{algorithm}
\end{frame}

\section{Síntesis de Voz}

\begin{frame}
  \Huge
  Síntesis de Voz
\end{frame}

\begin{frame}\frametitle{Síntesis de voz con SoundPlay}
  \begin{itemize}
  \item Es un paquete que permite reproducir archivos \texttt{.wav} o \texttt{.ogg}, sonidos predeterminados y síntesis de voz.
  \item La síntesis de voz se hace utilizando Festival (\url{http://www.cstr.ed.ac.uk/projects/festival/}).
  \item Para sintetizar voz, basta con correr el nodo \texttt{soundplay\_node} y publicar un mensaje de tipo \texttt{sound\_play/SoundRequest} con lo siguiente:
    \begin{itemize}
    \item msg\_speech.sound   = -3                 
    \item msg\_speech.command = 1                  
    \item msg\_speech.volume  = 1.0                
    \item msg\_speech.arg2    = ``voz a utilizar''
    \item msg\_speech.arg = ``texto a sintetizar''
    \end{itemize}
  \end{itemize}
\end{frame}

\begin{frame}[containsverbatim]\frametitle{Ejercicio 9}
  Ejecute el comando:
  \begin{lstlisting}
    roslaunch bring_up speech_synthesis.launch
  \end{lstlisting}
  En otra terminal, ejecute el comando:
  \begin{lstlisting}
    rosrun exercises speech_synthesis.py "my first synthetized voice"
  \end{lstlisting}
  Para instalar más voces:
  \begin{itemize}
  \item Ejecute el comando \texttt{sudo apt-get install festvox-<voz deseada>}
  \item Para ver qué voces se tienen instaladas: \texttt{ls /usr/share/festival/voices/english/}
  \end{itemize}
\end{frame}

\begin{frame}[containsverbatim]\frametitle{Ejercicio 9}
  Modifique el archivo \texttt{catkin\_ws/src/exercises/scripts/speech\_synthesis.py} y cambie la voz a utilizar en el mensaje SoundRequest.
  \lstinputlisting[language=Python,firstnumber=17]{Codes/SpeechSynthesis.py}
  El nombre de la voz se compone de \texttt{voice\_} más el nombre que aparece en la carpeta \texttt{/usr/share/festival/voices/english/}. 
\end{frame}

\section{Reconocimiento de voz}

\begin{frame}
  \Huge
  Reconocimiento de voz
\end{frame}

\begin{frame}\frametitle{Reconocimiento de voz con Pocketsphinx}
  Pocketsphinx es un \textit{toolkit} open source desarrollado por la Universidad de Carnegie Mellon (\url{https://cmusphinx.github.io/}).
  \begin{itemize}
  \item Aunque el toolbox original no está hecho específicamente para ROS, ya existen varios repositorios con nodos ya implementados que integran ROS y Pocketsphinx:
    \begin{itemize}
    \item \url{https://github.com/mikeferguson/pocketsphinx}
    \item \url{https://github.com/Pankaj-Baranwal/pocketsphinx}
    \end{itemize}
  \item El usuario debe estar agregado al grupo \textit{audio} para el correcto funcionamiento: \texttt{sudo usermod -a -G audio <user\_name>}
  \end{itemize}
  \begin{itemize}
  \item Se puede hacer reconocimiento usando una lista de palabras, un modelo de lenguaje o una gramática.
  \item Se utilizarán gramáticas y sus correspondientes diccionarios.
  \item Para construir diccionarios, visitar \url{https://cmusphinx.github.io/wiki/tutorialdict/}
  \item Para construir gramáticas, visitar \url{https://www.w3.org/TR/2000/NOTE-jsgf-20000605/}
  \end{itemize}
\end{frame}

\begin{frame}\frametitle{Ejercicio 10}
  \begin{enumerate}
  \item Verifique el volumne del micrófono
  \item Inspeccione el archivo \texttt{catkin\_ws/src/pocketsphinx/vocab/gpsr.gram} para ver las frases que se pueden reconocer de acuerdo con la gramática.
  \item Ejecute el comando \texttt{roslaunch bring\_up speech\_recognition.launch}
  \item En otra terminal, ejecute el comanto \texttt{rostopic echo /recognized}
  \item Pruebe el reconocimiento de voz con alguna de las siguientes frases:
    \begin{enumerate}
    \item \texttt{Robot, take the pringles to the table}
    \item \texttt{Robot, take the drink to the table}
    \item \texttt{Robot, take the pringles to the kitchen}
    \item \texttt{Robot, take the drink to the kitchen}
    \end{enumerate}
  \end{enumerate}
\end{frame}

\section{Planeación de acciones}

\begin{frame}\frametitle{Máquinas de estados}
  
\end{frame}

\bibliographystyle{abbrv}
\bibliography{References}
\begin{frame}
  \Huge{Gracias}
  \[\]
  \Large{Contacto}
  \[\]
  \large
  Dr. Marco Negrete\\
  Profesor Asociado C\\
  Departamento de Procesamiento de Señales\\
  Facultad de Ingeniería, UNAM.
\[\]
mnegretev.info\\
marco.negrete@ingenieria.unam.edu\\
\end{frame}
\end{document}
