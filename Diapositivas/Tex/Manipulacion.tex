
\begin{frame}\frametitle{Posición y Orientación}
  Un cuerpo rígido en el espacio puede tener una posición $(x,y,z)$ y una orientación. La orientación se puede representar de varias formas:
  \begin{itemize}
  \item Mediante ángulos de Euler: roll, pich y yaw $RPY = (\psi, \theta, \phi)$
  \item Mediante cuaterniones
  \item Mediante una matriz de rotación $R \in SO(3)$
  \end{itemize}
  Los ángulos $RPY$ son rotaciones intrínsecas sobre los ejes $X$, $Y$, y $Z$ respectivamente. Se llaman intrínsecas porque son rotaciones que ocurren sobre un sistema de referencia \textit{atado} a un cuerpo rígido.\\
  Cualquier orientación se puede obtener mediante la composición de tres rotaciones básicas:
  \[R = R_{z,\phi}R_{y,\theta}R_{x,\psi}\]
  Es decir, primero una rotación de $\phi$ radianes sobre el eje $Z$, seguida de una rotación de $\theta$ radianes sobre el eje $Y$ del sistema resultante y una rotación de $\psi$ radianes sobre el eje $X$ del sistema rotado. 
\end{frame}

\begin{frame}\frametitle{Transformaciones Homogéneas}
  Una Transformación Homogénea es una matriz de la forma
  Propiedades de cancelación de índices. 
\end{frame}
