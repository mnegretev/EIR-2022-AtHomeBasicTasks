\section{Visión}
\begin{frame}\frametitle{Las imágenes como funciones}
Una imagen (en escala de grises) es una función $I(x,y)$ donde $x,y$ son variables discretas en coordenadas de imagen y la función $I$ es intensidad luminosa.

  Las imágenes también pueden considerarse como arreglos bidimensionales de números entre un mínimo y un máximo (usualmente 0-255).

  Aunque formalmente una imagen es un mapeo $f:\mathbb{R}^2\rightarrow \mathbb{R}$, en la práctica, tanto $x,y$ como $I$ son varialbes discretas con valores entre un mínimo y un máximo.

  Las imágenes de color son funciones vectoriales $f:\mathbb{R}^2\rightarrow \mathbb{R}^3$ donde cada componente de la función se llama canal:
  \[I(x,y) = \left[\begin{tabular}{c}$r(x,y)$\\$g(x,y)$\\$b(x,y)$\end{tabular}\right]\]  
\end{frame}

\begin{frame}\frametitle{Espacios de color}
  
\end{frame}

\begin{frame}\frametitle{Histograma de color}
\end{frame}

\begin{frame}\frametitle{Nubes de puntos}
  
\end{frame}

\begin{frame}\frametitle{Extracción de planos}
  
\end{frame}


\begin{frame}\frametitle{Agrupamiento}
  
\end{frame}

\begin{frame}\frametitle{Ejercicio}
Quitar plano de la mesa y reconocer por color.   
\end{frame}

\begin{frame}\frametitle{Redes neuronales}
  
\end{frame}


\begin{frame}\frametitle{Ejercicio}
  
\end{frame}
